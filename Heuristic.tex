\documentclass[11pt]{article}

\usepackage[top=1in, bottom=1in, left=1in, right=1in]{geometry}
\usepackage{amsfonts}
\usepackage{graphicx}
\usepackage{float}
\usepackage[utf8]{inputenc}
\usepackage[section]{placeins}
\usepackage{amsmath}
\usepackage{amsfonts}
\usepackage{amssymb}
\usepackage{bm}
\usepackage[space]{grffile}

\begin{document}

\title{Neighbourhood Heuristic for Monkey Music Challenge}
\author{Sebastian Ånerud (910407-5958) \\
		anerud@student.chalmers.se}
\date{\today}
\maketitle

\begin{flushleft}

Let $m$ be the agents position, let $i$ denote an item in the set of items $I$ and $c$ be the number of items that your inventory can hold at most. Let $N_k(i)$ define the neighbourhood of size $k$ around $i$ including $i$. Also, let:

\begin{align*}
D_k(i) &= \sum\limits_{i'\in N_k(i)} d(i,i') \\
P_k(i) &= p(i) + \sum\limits_{i'\in N_k(i)} p(i'),
\end{align*}

where $d(i,i')$ is the shortest distance between $i$ and $i'$ and $p(i)$ is the number of point that item $i$ gives. Now define the value of a user $u$ as:

$$V(u) = \frac{1}{|I|} \sum\limits_{i \in I} \frac{d(u,i) + D_c(i)}{P_c(i)}$$

The heuristic for the pair ($N_k(i), u$) is then defined as:

\begin{align*}
H(N_k(i),u) = \frac{d(m,i) + D_k(i) + d(i,u)}{P_k(i) + p(CI)} + V(u)
\end{align*}

were $p(CI)$ is the number of points the items in your inventory are worth. Then define:

$$H_k^* = \min\limits_{i,u} H(N_k(i),u)$$
$$H^* = \min\limits_{k} H_k^*$$

Then chose the $i,u,k$ which minimizes $H^*$.
\end{flushleft}
\end{document}

